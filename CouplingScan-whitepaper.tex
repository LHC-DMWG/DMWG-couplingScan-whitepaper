\documentclass[a4paper, 11pt,notoc]{article}
\pdfoutput=1
\usepackage{jcappub}
\usepackage{graphicx}
\usepackage{booktabs}
\usepackage{verbatim}
\usepackage{caption}
\usepackage{xspace}
\usepackage{hyperref}
\usepackage{multirow}
\usepackage{placeins}
\usepackage{array}
\usepackage{subcaption}

% Variables
\newcommand{\sigv}{\ensuremath{\langle \sigma v_{\rm{rel}} \rangle}\xspace}
\newcommand{\MET}{\ensuremath{E_T^\mathrm{miss}}\xspace}
\newcommand{\met}{\MET}
\newcommand{\MT}{\ensuremath{M_{T}}\xspace}
\newcommand{\pt}{\ensuremath{p_{T}}\xspace}

% Units
\newcommand{\GeV}{\textrm{GeV}\xspace}
\newcommand{\gev}{\GeV\xspace}
\newcommand{\TeV}{\textrm{TeV}\xspace}
\newcommand{\tev}{\TeV\xspace}

% Particle names
\newcommand{\lp}{\ensuremath{l^{+}}\xspace}
\newcommand{\lm}{\ensuremath{l^{-}}\xspace}
\newcommand{\ttbar}{\ensuremath{\bar{t}t}}
\newcommand{\bbbar}{\ensuremath{\bar{b}b}}
\newcommand{\A}{A}
\newcommand{\pa}{a}
\newcommand{\pH}{H}
\newcommand{\pZ}{Z}
\newcommand{\Hc}{\ensuremath{H^{\pm}}\xspace}

% Particle masses
\newcommand{\mDM}{\ensuremath{M_{\chi}}\xspace}
\newcommand{\mdm}{\ensuremath{M_{\chi}}\xspace}
\newcommand{\mmed}{\ensuremath{M_{\rm{med}}}\xspace}
\newcommand{\mMed}{\ensuremath{M_{\rm{med}}}\xspace}
\newcommand{\mZ}{\ensuremath{M_{\rm{Z}}}\xspace}
\newcommand{\mA}{\ensuremath{M_{A}}\xspace}
\newcommand{\ma}{\ensuremath{M_{a}}\xspace}
\newcommand{\mH}{\ensuremath{M_{H}}\xspace}
\newcommand{\mHc}{\ensuremath{M_{H^{\pm}}}\xspace}
\newcommand{\mh}{\ensuremath{M_{h}}\xspace}
\newcommand{\mt}{\ensuremath{M_{t}}\xspace}

% Couplings
\newcommand{\gDM}{\ensuremath{g_{\chi}}\xspace}
\newcommand{\gq}{\ensuremath{g_q}\xspace}
\newcommand{\gl}{\ensuremath{g_l}\xspace}
\newcommand{\gdm}{\gDM}
\newcommand{\ifb}{\ensuremath{\rm{fb}^{-1}}\xspace}

% Other parameters
\newcommand{\sinp}{\ensuremath{\sin\theta}\xspace}
\newcommand{\cosp}{\ensuremath{\cos\theta}\xspace}
\newcommand{\sinbma}{\ensuremath{\sin(\beta - \alpha)}\xspace}
\newcommand{\cosbma}{\ensuremath{\cos(\beta - \alpha)}\xspace}
\newcommand{\tanb}{\ensuremath{\tan\beta}\xspace}
\newcommand{\lap}[1]{\lambda_{P#1}} % can use like \lap1 , \lap2
\newcommand{\lam}[1]{\lambda_{#1}} % can use like \lam3

% Search channels
\newcommand{\metplusx}{\ensuremath{\MET+X}\xspace}
\newcommand{\hdm}{\ensuremath{h+\textrm{DM}}\xspace}
\newcommand{\monoh}{\ensuremath{h+\MET}\xspace}
\newcommand{\monohbb}{\ensuremath{h(bb)+\MET}\xspace}
\newcommand{\monoz}{\ensuremath{Z+\MET}\xspace}
\newcommand{\monozll}{\ensuremath{Z(\ell\ell)+\MET}\xspace}
\newcommand{\monozhad}{\ensuremath{Z(\textrm{had})+\MET}\xspace}

% Software/Program/Model names
\newcommand{\mg}{\textsc{MadGraph~5}\xspace}
\newcommand{\mgamcnlo}{MG5\_aMC@NLO\xspace}
\newcommand{\dmsimp}{\textsc{DMsimp}\xspace}
\newcommand{\maddm}{\textsc{MadDM}\xspace}
\newcommand{\hdma}{\ensuremath{\textrm{2HDM+a}}\xspace}

% Misc
\newcommand{\GamA}{\ensuremath{\Gamma_{A}}\xspace}
\newcommand{\Uli}{\color{red}}
\definecolor{cerulean}{RGB}{44,150,207}
\newcommand{\ATLASComments}{\color{cerulean}}
\newcommand{\bra}[1]{\langle #1|}
\newcommand{\ket}[1]{|#1\rangle}
\newcommand{\sens}{\mathcal{S}\xspace}
\newcommand{\senstot}{\mathcal{S}_\textrm{tot}\xspace}
\newcommand{\mathsc}[1]{\text{\textsc{#1}}}
\newcommand{\vev}[1]{\langle {#1} \rangle}

\DeclareMathOperator{\arccot}{arccot}

\def\be   {\begin{equation}}   \def\ee   {\end{equation}}
\def\ba   {\begin{array}}      \def\ea   {\end{array}}
\def\bea  {\begin{eqnarray}}   \def\eea  {\end{eqnarray}}
\def\bean {\begin{eqnarray*}}  \def\eean {\end{eqnarray*}}
\def\nn{\nonumber}


\allowdisplaybreaks

%DIF PREAMBLE EXTENSION ADDED BY LATEXDIFF
%DIF UNDERLINE PREAMBLE %DIF PREAMBLE
\RequirePackage[normalem]{ulem} %DIF PREAMBLE
\RequirePackage{color}\definecolor{RED}{rgb}{1,0,0}\definecolor{BLUE}{rgb}{0,0,1} %DIF PREAMBLE
\providecommand{\DIFadd}[1]{{\protect\color{blue}\uwave{#1}}} %DIF PREAMBLE
\providecommand{\DIFdel}[1]{{\protect\color{red}\sout{#1}}}                      %DIF PREAMBLE
%DIF SAFE PREAMBLE %DIF PREAMBLE
\providecommand{\DIFaddbegin}{} %DIF PREAMBLE
\providecommand{\DIFaddend}{} %DIF PREAMBLE
\providecommand{\DIFdelbegin}{} %DIF PREAMBLE
\providecommand{\DIFdelend}{} %DIF PREAMBLE
%DIF FLOATSAFE PREAMBLE %DIF PREAMBLE
\providecommand{\DIFaddFL}[1]{\DIFadd{#1}} %DIF PREAMBLE
\providecommand{\DIFdelFL}[1]{\DIFdel{#1}} %DIF PREAMBLE
\providecommand{\DIFaddbeginFL}{} %DIF PREAMBLE
\providecommand{\DIFaddendFL}{} %DIF PREAMBLE
\providecommand{\DIFdelbeginFL}{} %DIF PREAMBLE
\providecommand{\DIFdelendFL}{} %DIF PREAMBLE
%DIF END PREAMBLE EXTENSION ADDED BY LATEXDIFF

\def\bm#1{\mbox{\boldmath$#1$\unboldmath}} 

\begin{document}
\title{\begin{boldmath} \huge Displaying dark matter constraints from colliders with varying simplified model parameters \vspace{7mm} \end{boldmath}}

%%%%%%

%Add your name here!

\author[1]{Andreas~Albert,}
\affiliation[1]{Boston University, Dept.\ of Physics, 590 Commonwealth Avenue, Boston, \\ MA 02215, USA}

\author[2]{Antonio~Boveia,}
\affiliation[2]{Ohio State University and Center for Cosmology and Astroparticle Physics, \\
191 W. Woodruff Avenue Columbus, OH 43210, USA}

\author[3,*]{Oleg~Brandt,}
\affiliation[3]{Cavendish Laboratory, JJ Thomson Avenue, Cambridge CB3 0HE, UK}

\author[4]{Eric~Corrigan,}
\affiliation[4]{Fysiska institutionen, Lunds universitet, Professorsgatan 1, Lund, Sweden}

\author[1]{Zeynep~Demiragli,}

\author[4]{Caterina~Doglioni,}

% KP: i specifically added Etienne bc I used some of his dilepton stuff, pls keep
\author[5]{Etienne Dreyer,}
\affiliation[5]{Weizmann Institute of Science, Herzl St 234, Rehovot, Israel}

\author[6]{Boyu~Gao,}
\affiliation[6]{Duke University, Durham, NC 27708, USA}

\author[7,*]{Ulrich~Haisch,}
\affiliation[7]{Max Planck Institute for Physics, F{\"o}hringer Ring 6,  80805 M{\"u}nchen, Germany}

\author[8,*]{Philip~Harris,}
\affiliation[8]{Massachusetts Institute of Technology, 77 Massachusetts Avenue, Cambridge, MA, USA}

\author[]{Jeffrey~Krupa,} % KP: who is this?

\author[9]{Greg~Landsberg,} % KP: who is this?
\affiliation[9]{Brown University, Dept.\ of Physics, 182 Hope St, Providence, RI 02912, USA}

\author[10]{Alexander~Moreno,}
\affiliation[10]{Universidad Antonio Nari\~{n}o, Bogotá, Cundinamarca, Colombia}

\author[11]{Katherine~Pachal,}
\affiliation[11]{Duke University, Durham, NC 27708, USA}

\author[12,*]{Priscilla~Pani,}
\affiliation[12]{DESY Zeuthen, Platanenallee 6, 15738 Zeuthen, Germany}

% KP: i specifically added Federica bc she provided me with all the monophoton help I needed in developing the mono-x method, pls keep
\author[13]{Federica~Piazza,}
\affiliation[13]{INFN Milano, Via G. Celoria, 16 I - 20133 Milano, Italy}

\author[14,*]{Tim~M.~P.~Tait,}
\affiliation[14]{University of California Irvine, Irvine, CA 92697, USA}

\author[9]{David~Yu,}

\author[15]{Felix~Yu,}
\affiliation[15]{Institut für Physik WA THEP, Johannes Gutenberg-Universität Mainz, Staudingerweg 7, 55128 Mainz, Germany}

\author[16]{Lian-Tao~Wang}
\affiliation[16]{Department of Physics, University of Chicago, Chicago, IL. 60637 USA}

\affiliation[*]{LHC DM WG conveners}


\hfill CERN-LPCC-2020-XX

\abstract{
Dark matter is one of the main science drivers of the particle and astroparticle physics community.  Determining the nature of dark matter will require a broad approach, with a range of experiments pursuing different experimental hypotheses. As one axis of this search program, collider experiments provide insight on dark matter complementary to searches in direct/indirect detection experiments and to astrophysical evidence.

In order to compare results from a wide variety of experiments, a common theoretical framework is required. Among the numerous theoretical frameworks that describe dark matter (see e.g.~\cite{doi:10.1146/annurev-astro-082708-101659}, and \cite{Kahlhoefer:2017dnp,doi:10.1146/annurev-nucl-101917-021008} for collider-focused reviews), the ATLAS and CMS experiments at the Large Hadron Collider have adopted a series of simplified models that include dark matter particles as benchmarks for their searches.

In these models, the interaction between Standard Model (SM) and dark matter particles is mediated by a new particle, called a mediator. The interaction strength is controlled by the couplings of the mediator to dark matter and to SM particles. 

So far, the presentation of LHC results (as well as the presentation of projections of future hadron collider experiments) has focused on four benchmark scenarios with different choices of couplings to quarks and leptons, as recommended by the Dark Matter Working Group~\cite{BOVEIA2020100365, ALBERT2019100377}.

In this work, we describe methods to extend those four benchmark scenarios to scenarios with arbitrary couplings, and release the corresponding code for use in further studies and projections of collider dark matter searches in the framework of simplified models. This will extend the applicability of the comparisons of collider searches to accelerator experiments that are sensitive to smaller couplings, and give a more complete picture of the coupling dependence of the sensitivity of dark matter collider searches when compared to direct and indirect detection searches. By using semi-analytical methods to model the dependence, we plan to drastically reduce the need for computing resources relative to traditional approaches based on the generation of additional simulated signal samples.

We focus here on s-channel DM simplified models where the mediator particle has vector or axial-vector couplings to DM and SM particles. This work will cover collider searches for visible decays of the mediator particles, as well as for searches targeting the invisible particles via the associated production of one or more SM particles~\cite{ATL-PHYS-PUB-2020-021,CMSSummary}.
}  

\maketitle

%\newpage 


\vskip10pt


%\bibliography{CouplingScan-whitepaper}
%\bibliographystyle{JHEP}

%\end{document}

%Here ends the LOI version

%%%%%%%%%%%%%%%%%%%%%%%%%%%%%%%%%%%%%%%%%%%%%%%%%%%%%%%%%%%%%%%%%%%%
%%%%%%%%%%%%%%%%%%%%%%%%%%%%%%%%%%%%%%%%%%%%%%%%%%%%%%%%%%%%%%%%%%%%
%%%%%%%%%%%%%%%%%%%%%%%%%%%%%%%%%%%%%%%%%%%%%%%%%%%%%%%%%%%%%%%%%%%%

\section{Introduction}
\label{sec:introduction}

The search for new particles belonging to dark sectors is one of the primary focuses of the particle physics community, both at dedicated dark matter search experiments and at colliders. The ATLAS and CMS collaborations have focused Since the establishment by the LHC Dark Matter Working Group of a specific set of simplified models in 20XX, ATLAS and CMS have used these models as the basis for many of their statements on dark matter exclusions. These simplified models, while they are only one framework for interpreting dark matter limits and should be complemented by e.g. 2HDM+a and SUSY based models, serve an important purpose by \ldots

% Nice paragraph on complementarity and how this will let us probe it

The goal of this paper is to establish techniques for converting analysis limits computed in the context of one set of simplified model parameters into any other desired set of model parameters, with reasonable accuracy and without resorting to generating events. This will allow 



While the techniques shown here are extendable to lepton colliders, they have only so far been developed in the context of hadron colliders. All results presented today are based on LHC limits, and the software published in conjunction with this paper currently only serves hadronic initial states. 

Note on how this is for Snowmass and a follow-up version will be submitted for publication; it will contain fully documented and supported code package. For now, code is available on Github here.

%%%%%%%%%%%%%%%%%%%%%%%%%%%%%%%%%%%%%%%%%%%%%%%%%%%%%%%%%%%%%%%%%%%%
\section{Models considered}
\label{sec:models}

The models considered in this study are detailed extensively in Refs.~\cite{ABERCROMBIE2020100371,Albert:2017onk}. Each simplified model adds two new particles beyond the Standard Model: a Dirac fermion dark matter particle $\chi$ and an $s$-channel mediator particle. This mediator can be spin-1, in which case its couplings can be either vector or axial-vector in nature, or spin-0, in which case its couplings are scalar or pseudo-scalar. This Snowmass paper draft only provides rescaling formulas for the models with vector or axial-vector mediators. For resonant final states like dijet and dilepton, there is no strong motivation to extend the rescaling methods developed here, this since these final states do not provide a strong constraint on scalar and pseudoscalar models. For invisible final states it would indeed be very useful, however the different leading order processes in these models mean that a similar approach to rescaling couplings will require separate derivation.  This will be left as a project for a future paper. Further discussion of the models here will therefore focus only on the vector and axial-vector scenarios.

% Talk about couplings independence, what can be set, what vertices we have
The coupling of the vector and axial-vector mediators to Standard Model particles is generally treated as  This leads to a model fully defined by five free parameters: the mediator mass \mMed, the dark matter particle mass \mDM, the coupling strength of a vertex between the mediator and two dark matter particles \gdm, a single universal coupling between the mediator and Standard Model quarks \gq, and a single universal coupling between the mediator and Standard Model leptons \gl. For lighter mediator masses, decays into heavy quarks or leptons will be kinematically forbidden, but the coupling is always taken to be constant for all fermion generations.

% Mention leading order diagrams and our key analyses.
The leading order processes for 

% Mention relevance of going to NLO for real limits but possibility of rescaling at LO - check that out, but our calculations here are tree-level


%%%%%%%%%%%%%%%%%%%%%%%%%%%%%%%%%%%%%%%%%%%%%%%%%%%%%%%%%%%%%%%%%%%%
\section{Resonant final states}
\label{sec:resonant}

Why we can start from 1D limit here: tie it into physical signature. Mention that math worked out in this section assumes you have negligible sensitivity to intrinsic width; see next section for otherwise.

Point to cross-section-mass plot. Exclusion depth is obs/theory - thus in 1D exclusion limit is where the two cross. For mass-mass plots this depth exists in 2D plane and what we want is twofold: find it from inputs that are standardly produced by analysis teams and 2) rescale it analytically to other couplings etc.

% Introduce: this was developed by CMS and has been in use for a long time.  Taking this opportunity to document it in detail.

% Model relevance - vector/AV only, because di-x is not a strong constraint on scalar/pseudoscalar models, so we don't need to worry about it

What's fixed? Can't rescale from one mediator mass to another since that is what directly dictates analysis properties. So can only interpret in other dimensions.
Can interpret across dark matter mass, specifically because no DM mass present in leading order diagram for visible final states.
Can interpret across couplings, since changing couplings changes cross sections and widths but nothing else.

Approach: change cross section of signal as long as it stays within bounds where the experimental limit for such a model would be the same, or is otherwise known. That is, shape and/or acceptance do not change outside the bounds of what we have experimental results to cover. In this case, can rescale the exclusion depth by the ratio of theoretical cross sections for the scenario in which the limit plot was made and the scenario in which you wish to interpret it. We use a range of approximations for those cross sections depending on 

Comment on assumptions here? Probably?

\subsection{Dijet and dijet+X}


% Goal: begin from standard final plot, and/or g	q plot.

% On-shell decay to quarks is possible everywhere, so we can use on-shell approximations everywhere. At tree level just the one diagram with two quark couplings. Explain how ISR factors out so it will work fine for dijet+ISR too.

% How pt 1: cross section limit to 2d limit

% How pt 2: rescaling the 2d limit (super easy)

% Assumptions: no change in acceptance across those points. And narrow width (inbuilt in approximation, comment). K-factor? Or is this order-independent?
The primary assumption required by this approach is that the analysis acceptance is constant for a single mediator mass, regardless of dark matter mass and coupling choices. This assumption is reasonable so long as the intrinsic width of the mediator is narrower than the analysis mass resolution, but should be checked for each analysis and the boundaries of its validity well understood. In cases where the mass resolution is good and this assumption cannot be fairly made, e.g. dilepton final states, a modified technique can be applied as described in Section~\ref{sec:dilepton}. 

Two additional assumptions are also required. First, the mediator must have a sufficiently narrow intrinsic width that [....]. {\color{red} can we quantify when this fails? Likely safe given that it falls inside the previous assumption, but still....}. Second, the $k$-factors are assumed to be equal at all points independent of DM mass and of coupling for a given mediator mass.

% Illustrate with a plot? Could show from the one CMS one that hasn't been interpreted yet, that I can get a good mediator plot using non-gdm=0 starting?
 

\subsection{Dilepton and other high-resolution resonances}
\label{sec:dilepton}

% CMS dilepton already doing this https://arxiv.org/pdf/1803.06292.pdf, could show something from it as an example

Discuss the di-lepton final state requires accounting for widths more carefully since the resolution is much better. Point out this method also works for eliminating assumptions about width/acceptance consistency in the dijet limit case - can use the same code there. 

This removes the need to assume that the acceptance is unchanged as a function of dark matter mass and coupling values. Instead, by providing a range of limits corresponding to various intrinsic width to mass ratios for a single mediator mass point, the most appropriate one can be chosen for each coupling and DM mass.

% Interpolation we are doing in between them - not yet settled. I am just thinking linear, and no limit given if wider than the widest line?
There are several possible ways to approach points with widths in between the provided curves. The most conservative approach would always use the weaker of the two neighbouring limit curves (that is, the one corresponding to larger width). To produce a smoothly varying exclusion result instead of one with distinct regions corresponding to transitions between the curves. The current implementation uses a linear interpolation between the limit curves to select constraints on intermediate widths, \ldots [and what above and below?]


%%%%%%%%%%%%%%%%%%%%%%%%%%%%%%%%%%%%%%%%%%%%%%%%%%%%%%%%%%%%%%%%%%%%
\section{Mono-$X$ final states}
\label{sec:monox}

% Why mono-x signatures need to contend with off-shell scenarios and thus cannot use the same approximations
For \metplusx signatures, the relevant final state involves decay of a mediator to dark matter particles, and therefore the off-shell case has to be handled appropriately.
The approximations used for visible resonant final states in Section~\ref{sec:resonant} require a well-defined decay width to the final state $\Gamma_f$, which goes to zero at $\mMed=2\mDM$ for an invisible final state. The approximation in fact loses validity before this diagonal is fully reached, as the transition across the on-shell to off-shell boundary is smooth. A method for re-scaling \metplusx signatures has therefore been developed with the goal of handling this transition smoothly such that it is applicable in all regimes.

% How have they been handled previously? Discuss method.
Previously, ATLAS and CMS \metplusx analyses have generated a grid of signal points in one of the four benchmark scenarios at full reconstruction level and used these to determine the region of \mMed-\mDM space excluded in that scenario~\cite{atlas_monojet_36ifb, cms_monojet_12ifb}. Three additional grids of signal points are then generated at leading order and particle level in order to obtain cross section estimates for each point in the additional three benchmark scenarios. The ratio of cross sections between the points in the original scenario and the target scenarios is used to scale the limits, resulting in a new estimate of the excluded and non-excluded points within the target scenario. 
Using signal generation to determine cross sections creates a time and CPU limitation on the number of scenarios which can be explored. 
% What do we begin from? Mass-mass exclusion plane in one of our models for some fixed set of parameter values
Therefore the goal of these studies is to use the same starting information - a full signal grid in the \mMed-\mDM for one coupling scenario - and determine a method for rescaling to another target scenario without requiring the generation of many individual signal points. 

% What assumptions do we make? Assume acceptances don't change as a result of varying couplings; 
% assume k-factors are essentially flat across this plane such that LO rescaling of NLO x-section arrives at fairly decent NLO x-sec.
% Do we need to back these up? For now, point out that same set of assumptions as with existing method.
Two assumptions are commonly made when applying the existing technique: that first, the $k$-factors relating LO to NLO cross sections, and second, the fiducial acceptance of an analysis are both essentially invariant with coupling for fixed masses. The consistency of $k$-factors has been verified by generating NLO signals for a subset of the relevant scenarios and implies that a ratio of LO cross sections applied to an NLO cross section in the original scenario provides an adequate approximation of an NLO cross section in the target scenario. Invariance of analysis acceptance was verified in the $\met+j$ and $\met+\gamma$ analyses and means no additional factor is required to account for changes in mediator width~\cite{DMSP}.
The method developed here keeps these same assumptions, producing a re-scaling based only on LO cross sections with no consideration for experimental acceptance. It is recommended that a user consider the validity of these assumptions when applying the new rescaling techniques. 

The semi-analytical rescaling method developed here for \metplusx signatures has two separate components. One can be used to rescale a source scenario to another set of couplings within the same overall model, while the other should be used when the source scenario in one model is rescaled to a target in a different model (e.g. vector mediator to axial-vector mediator).

% Cite original paper saying approximations have been given before for scaling on and off-shell, but that we make the simple extension of considering the integral of the entire propagator.
In the original whitepaper defining the simplified models studied here, it is specified that the cross section scaling can be estimated using the integral of the Breit-Wigner propagator for the mediator~\cite{Abercrombie:2015wmb}. Several approximations of this integral are given, corresponding to the different regimes of on-shell mediators, off-shell mediators, and effective field theories. In order to smoothly handle the on-shell to off-shell transition, we instead use the full integral of the propagator over the permitted phase space $s \geq 4 \mDM^2$:
\begin{align}
\mathcal{I}_{\text{prop}} =&\  g_q^2 g_\chi^2 \int_{4\mDM^2}^{\infty} \frac{ds}{(s-\mMed^2)^2+\mMed^2\Gamma^2} \\
=&\  \frac{g_q^2 g_\chi^2}{\Gamma\mMed} \left(\frac{\pi}{2} + \arctan{\left(\frac{\mMed^2 - 4 \mDM^2}{\Gamma\mMed}\right)} \right)\,.
\end{align}
% Within a single model: integral of the propagator as sufficient for demonstrating scaling
Coupling dependence comes not only from the explicit $g_q^2 g_\chi^2$ factor but is also embedded in the value of $\Gamma$ (see width definitions in Ref.~\cite{Albert:2017onk}).
$\mathcal{I}_{\text{prop}}$ is easily analytically calculable and the ratio of its value for two different coupling scenarios at a given \mDM and \mMed provides a robust estimate of the cross section in the target scenario, so long as both assume the same mediator type.

% Show example
The $\mathcal{I}_{\text{prop}}$-based scaling procedure is illustrated in Figure~\ref{fig:propagator_scaling} using the observed limits from the ATLAS $\met+\gamma$ analysis in 36 \ifb of data~\cite{monophoton}. Circles show the locations of the signal points used. Each signal point has a $z$-axis value of theoretical cross section divided by excluded cross section; white points are those with $z<1$ and are excluded while red points have $z>1$ and are not excluded. The red curves show the exclusion contours reported by the analysis in each plane. The blue shades are a linear interpolation between the points used to illustrate the cross section exclusion surface. In Figure~\ref{subfig:monophoton_A1} no reinterpretation is done: the values at each point are taken directly from the paper and overlaid with the corresponding exclusion curve. In Figure~\ref{subfig:monophoton_A2} the values at each point are calculated using the $\mathcal{I}_{\text{prop}}$ scaling procedure starting from the values in Figure~\ref{subfig:monophoton_A1}. The white and red colour coding of points is based on these rescaled values, as is the linear interpolation in blue. The good agreement between the contour curve taken from the paper and the points marked as excluded by the rescaling method serves as a validation of the rescaling.

%-----------------------------------
\begin{figure}[htp!]
  \begin{center}
  \begin{subfigure}[b]{0.49\textwidth}  
    \includegraphics[width=\textwidth]{figures/monox/EXOT-2016-32_AV_gq0p25_gchi1p0_gl0p0_compare.pdf}
    \caption{}
    \label{subfig:monophoton_A1}
  \end{subfigure}
  \begin{subfigure}[b]{0.49\textwidth}  
    \includegraphics[width=\textwidth]{figures/monox/EXOT-2016-32_AV_gq0p1_gchi1p0_gl0p1_compare.pdf}
    \caption{}
    \label{subfig:monophoton_A2}  
  \end{subfigure}
  
  \caption{Original cross section limits with couplings $g_q=0.25, g_\chi=1.0, g_l = 0.0$ (\subref{subfig:monophoton_A1}) and rescaled cross section limits with couplings $g_q=0.1, g_\chi=1.0, g_l = 0.1$ (\subref{subfig:monophoton_A2}) for an axial-vector mediator model using the ATLAS $\met+\gamma$ analysis results. Limits are presented as a function of mediator mass and dark matter mass for fixed coupling values. At each signal point, the value of $z = \sigma_{\text{theory}}/\sigma_{\text{excluded}}$ is calculated: points where $z<1$ are excluded and plotted in white, while points where $z>1$ are not excluded and plotted in red. A linear interpolation between the $z$ values at the points illustrates the shape of this exclusion surface and is shown as a blue gradient. The solid red curves are taken from the published $\met+\gamma$ results and show the exclusion contour $\sigma_{\text{theory}}/\sigma_{\text{excluded}} = 1$. The level of agreement between the excluded points and the published contour in~(\subref{subfig:monophoton_A2}) is a measure of the similarity in performance between the $\mathcal{I}_{\text{prop}}$ scaling procedure and the Monte Carlo based method used in the ATLAS analysis.}
  \label{fig:propagator_scaling}
  \end{center}
\end{figure}
%-----------------------------------

% Introduce that we need full cross section to go between models.
Rescaling cross sections using a ratio based on $\mathcal{I}_{\text{prop}}$ is no longer sufficient when the target scenario uses a different mediator model than the source scenario. Additional mass-dependent terms in the cross section vary between models, so while they cancel within a single model and can be ignored in favour of the simple propagator-based expression, they must be correctly accounted for when rescaling from one model to another. We therefore calculate a more complete cross section whose ratio can be used to perform across-model rescaling.

% Introduce that for cross sections at LO that amounts to only considering this one diagram
At leading order, the production of an $s$-channel vector or axial-vector mediator decaying to two dark matter particles includes just one diagram, $pp\rightarrow Z^\prime \rightarrow \chi \chi$. The LO signal contribution to each \metplusx analysis consists of that diagram with the additional ISR radiation of some object (jet, photon, etc) from one of the incoming quarks. We consider that the properties and probability of this radiation are independent of the couplings of the DM model and therefore act as a consistent scale factor on the cross section which will cancel in the ratio for a specified pair of masses \mMed, \mDM. Therefore, to calculate the LO scale factor translating between two scenarios with different mediator models, the cross section for that single diagram in both models is sufficient. The parton-level cross sections as functions of the interaction scale $S$ are given by the following equations, discounting scale factors which cancel in the ratio:

% Introduce full cross sections for vector, axial vector
\begin{equation}
\sigma_V(S) \propto \frac{g_q^2 g_\chi^2 (S + 2\mDM^2)\sqrt{S-4\mDM^2}}{(\Gamma^2\mMed^2 + (\mMed^2 - S)^2)\sqrt{S}}
\end{equation}
and
\begin{equation}
\sigma_{AV}(S) \propto \frac{g_q^2 g_\chi^2 (S-4\mDM^2)^{3/2}}{(\Gamma^2\mMed^2 + (\mMed^2 - S)^2)\sqrt{S}}\,,
\end{equation}
where $\Gamma$ is the total width of the mediator (non-zero even when decays to dark matter are off-shell).

% Discuss PDFs.
To calculate cross sections with sufficient accuracy for the rescaling procedure, a hadron-level quantity must be determined accounting for parton distribution functions (PDFs). The full cross section is thus calculated semi-analytically by performing a two dimensional integral over the longitudinal momentum fractions $x_1, x_2$ of the interacting partons within the allowed range. Where $S$ is the squared centre-of-mass energy (i.e. $\sqrt{S} = 13$ TeV), $\hat{s} = x_1 x_2 S$ is the scale of the interaction, and $f^a$ and $f^b$ are the PDFs of the incoming partons, the total cross section is
\begin{equation}
\sigma_{V/AV}^{\text{tot}} = \int_{4 \mDM^2}^{\text{S}} \int_{4 \mDM^2}^{x_2 S}  f^a(x_1,\hat{s})  f^b(x_2,\hat{s}) \sigma_{V/AV}(\hat{s}) dx_1 dx_2\,.
\end{equation}
This integral is performed numerically in the two models and the ratio at each point is taken as the scale factor to convert between them. The implementation uses LHAPDF and so provides a wide range of PDF sets and flavour schemes, although results are found to be fairly independent of the selection~\cite{Buckley:2014ana}.

% Show example scaling between models
A demonstration of the $\sigma_{V/AV}^{\text{tot}}$-based scaling procedure is given in Figure~\ref{fig:pdf_scaling}. The $z$-value at each point in Figure~\ref{subfig:monophoton_V2} is found using the ratio of the vector and axial-vector cross sections $\sigma_{V}^{\text{tot}}$ and $\sigma_{AV}^{\text{tot}}$ applied to the axial-vector limits from Figure~\ref{subfig:monophoton_A1}. The success of this rescaling method is again illustrated by the agreement between the excluded points (white) determined by the rescaling and the solid curve obtained from the analysis paper. Figure~\ref{subfig:monophoton_V2} is created by further applying a $\mathcal{I}_{\text{prop}}$ rescaling to the results in Figure~\ref{subfig:monophoton_V1}. Even after this iterative application of $\sigma_{V/AV}^{\text{tot}}$ and $\mathcal{I}_{\text{prop}}$ rescaling, the agreement between the points and the published results is fair.

%-----------------------------------
\begin{figure}[htp!]
  \begin{center}
  \begin{subfigure}[b]{0.49\textwidth}  
    \includegraphics[width=\textwidth]{figures/monox/EXOT-2016-32_V_gq0p25_gchi1p0_gl0p0_compare.pdf}
    \caption{}
    \label{subfig:monophoton_V1}
  \end{subfigure}
  \begin{subfigure}[b]{0.49\textwidth}  
    \includegraphics[width=\textwidth]{figures/monox/EXOT-2016-32_V_gq0p1_gchi1p0_gl0p01_compare.pdf}
    \caption{}
    \label{subfig:monophoton_V2}  
  \end{subfigure}
  
  \caption{Rescaled cross section limits with couplings $g_q=0.25, g_\chi=1.0, g_l = 0.0$ (\subref{subfig:monophoton_V1}) and $g_q=0.1, g_\chi=1.0, g_l = 0.01$ (\subref{subfig:monophoton_V2}) for a vector mediator model using the ATLAS $\met+\gamma$ analysis results. Limits are presented as a function of mediator mass and dark matter mass for fixed coupling values. At each signal point, the value of $z = \sigma_{\text{theory}}/\sigma_{\text{excluded}}$ is calculated: points where $z<1$ are excluded and plotted in white, while points where $z>1$ are not excluded and plotted in red. A linear interpolation between the $z$ values at the points illustrates the shape of this exclusion surface and is shown as a blue gradient. The solid red curves are taken from the published $\met+\gamma$ results and show the exclusion contour $\sigma_{\text{theory}}/\sigma_{\text{excluded}} = 1$. The level of agreement between the excluded points and the published contour is a measure of the performance of the $\sigma_{V/AV}^{\text{tot}}$ scaling procedure in~(\subref{subfig:monophoton_V1}) and of the combination of both procedures in~(\subref{subfig:monophoton_V2}).
  }
  \label{fig:pdf_scaling}
  \end{center}
\end{figure}
%-----------------------------------

% Summarise algorithm: use full cross section integral to translate initial plane into any other models desired. 
% Use propagator scaling to scan coupling exclusions within that model.
The full recommended procedure for rescaling \metplusx analysis results is therefore to use the $\sigma_{V/AV}^{\text{tot}}$-based rescaling procedure to convert from the original model with one type of mediator into a single scenario with the desired new mediator type (e.g. vector mediator to axial-vector mediator). The $\mathcal{I}_{\text{prop}}$-based rescaling procedure can then be used to convert to other coupling scenarios with the same mediator type.

%%%%%%%%%%%%%%%%%%%%%%%%%%%%%%%%%%%%%%%%%%%%%%%%%%%%%%%%%%%%%%%%%%%%
\section{Combined examples of coupling-scaled exclusion plots}

Would like to show a full summary plot with 36 ifb results from a variety of channels, then scale to at least one set of as-yet-unexplored couplings, to illustrate the power. Probably just 1 to 2 figures here.


%%%%%%%%%%%%%%%%%%%%%%%%%%%%%%%%%%%%%%%%%%%%%%%%%%%%%%%%%%%%%%%%%%%%
\section{Relic densities and their relationships to couplings in DM simplified models}

{\color{red}Put cool plots from Phil here}

%%%%%%%%%%%%%%%%%%%%%%%%%%%%%%%%%%%%%%%%%%%%%%%%%%%%%%%%%%%%%%%%%%%%
\section{Conclusion}

% We have demonstrated the way that you can mathematically rescale from AV to V and back, between different couplings, for invisible and visible final states whose LO diagrams are the ones given here.

% Certain caveats to do with acceptances - these are the responsibility of the reader

% Certain caveats to do with k-factors - can make no guarantees so check or sort out an extrapolation if concerned

% Code will be made public with full paper, but you can of course reach out to us for it now, and it is available to members of both ATLAS and CMS already

% Interested in a follow-up that will include scalar and pseudoscalar, but math is quite different.

%%%%%%%%%%%%%%%%%%%%%%%%%%%%%%%%%%%%%%%%%%%%%%%%%%%%%%%%%%%%%%%%%%%%
%%%%%%%%%%%%%%%%%%%%%%%%%%%%%%%%%%%%%%%%%%%%%%%%%%%%%%%%%%%%%%%%%%%%
%%%%%%%%%%%%%%%%%%%%%%%%%%%%%%%%%%%%%%%%%%%%%%%%%%%%%%%%%%%%%%%%%%%%

\acknowledgments 

[To be updated] The research of A.~Boveia is supported by the U.S. DOE grant  DE-SC0011726. C.~Doglioni has received funding from the European Research Council under the European Union's Horizon 2020 research and innovation program (grant agreement 679305) and from the Swedish Research Council. U.~Haisch acknowledges the hospitality and support of the CERN Theoretical Physics Department. The work of T.~M.~P.~Tait is supported in part by NSF grant PHY-1316792. We gratefully acknowledge the support by the U.S. DOE. 

%%%%%%%%%%%%%%%%%%%%%%%%%%%%%%%%%%%%%%%%%%%%%%%%%%%%%%%%%%%%%%%%%%%
%%%%%%%%%%%%%%%%%%%%%%%%%%%%%%%%%%%%%%%%%%%%%%%%%%%%%%%%%%%%%%%%%%%%
%%%%%%%%%%%%%%%%%%%%%%%%%%%%%%%%%%%%%%%%%%%%%%%%%%%%%%%%%%%%%%%%%%%%

% KP: for the "real" paper we should include an appendix documenting the code usage but for this stage we are not ready to document the code, so we don't need an appendix
% \appendix

% \section{Appendix}
% \label{app:recast}

% Document public code here?

\newpage 

\bibliography{CouplingScan-whitepaper}
\bibliographystyle{JHEP}



\end{document}
